\providecommand{\OrgLocTitleDate}[4]{
    \textbf{#1}             \hfill          #3\\
    \textsl{#2}             \hfill          \textsl{#4}}

\providecommand{\DatePoint}[2]{
    {#1}\hfill \textsl{#2}}


% If commands do not exist, initialise them
% \newcommand{name}[num]{definition}
\providecommand{\TitleSubtitleLocationDates}[4]{
    \textbf{#1}             \hfill          #3\\
    \textsl{#2}             \hfill          \textsl{#4}}

\providecommand{\optionA}{}
\providecommand{\optionB}{}
\providecommand{\optionC}{}
\providecommand{\optionCustom}{}

\providecommand{\optionRule}{{\color{black!20}\hrule width\textwidth}}

\providecommand{\textA}{}
\providecommand{\textB}{}
\providecommand{\textC}{}

\renewcommand{\textA}{}
\renewcommand{\textB}{}
\renewcommand{\textC}{}

% Now commands always exist, and are reset to the following
\renewcommand{\optionA}{
    \textcolor{black!20}{
    \vspace{0mm}
    \begin{itemize}
    \item This is an option to points in a description.
    \item option A will generally be long
    \end{itemize}
    }
    
}
\renewcommand{\optionB}{
    \textcolor{black!20}{Option B will generally be a smaller description}

}
\renewcommand{\optionC}{
    \textcolor{black!20}{Short description}

}
